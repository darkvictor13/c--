\documentclass[12pt]{beamer}
\usepackage[utf8]{inputenc}
\usepackage[portuguese]{babel}
\usepackage{graphicx}
\usepackage{colortbl}
\usepackage{color}
\usepackage{cite}
\usepackage{breqn}
\usepackage{listings}

\graphicspath{{./images}}

\definecolor{dkgreen}{rgb}{0,0.6,0}
\definecolor{gray}{rgb}{0.5,0.5,0.5}
\definecolor{mauve}{rgb}{0.58,0,0.82}
\definecolor{laranja_claro}{rgb}{1,0.9,0.5}
\definecolor{laranja_escuro}{rgb}{1,0.5,0.2}
\definecolor{azul_claro}{rgb}{0.5,0.9,1}

\lstset{frame=tb,
  language=C,
  aboveskip=3mm,
  belowskip=3mm,
  showstringspaces=false,
  columns=flexible,
  basicstyle={\small\ttfamily},
  numbers=left,
  numberstyle=\tiny\color{gray},
  keywordstyle=\color{blue},
  commentstyle=\color{dkgreen},
  stringstyle=\color{mauve},
  breaklines=true,
  breakatwhitespace=true,
  tabsize=4
}

\definecolor{azul}{rgb}{0,0,.5}
\setbeamertemplate{navigation symbols}{}

\usetheme{Frankfurt}
\usecolortheme[named=azul]{structure}

%% Definindo o Autor e o título
\newcommand{\prof}{Newton Spolaôr}
\newcommand{\materia}{Compiladores}

\author[Grupo: c--]{Victor E. Almeida \and Marco A. G. Pedroso}

\title{Analisador léxico para a linguagem C--}
\subtitle{Demonstrando o software}
\date{\today}
\institute{UNIOESTE}

\begin{document}
\frame{\titlepage}

\begin{frame}
\frametitle{Conteúdo}
\tableofcontents
\end{frame}

\section{Introdução}\label{Introdução}
\begin{frame}
    \frametitle{Softwares utilizados}
    \begin{itemize}
        \item \textbf{Sistema para compilar}: GNU make,
        \item \textbf{Linguagem de programação}: ANSI C,
        \item \textbf{Gerador de analisador léxico}: GNU flex.
    \end{itemize}
\end{frame}

\begin{frame}[t,fragile]{\insertsectionhead}
    \frametitle{Funcionamento do lex/flex}

    \begin{lstlisting}
    EXPRESSION_BEGIN "("
    {EXPRESSION_BEGIN} {
        doLog (
            LOG_TYPE_INFO,
            "inicio de expressao [(] encontrado"
        );
        is_open_expression++;
    }
    \end{lstlisting}
\end{frame}

\begin{frame}[t,fragile]{\insertsectionhead}
    \frametitle{Exemplo minimo do flex}
    \begin{lstlisting}
        int num_lines = 0, num_chars = 0;

        %%
        \n      ++num_lines; ++num_chars;
        .       ++num_chars;
        %%

        int main() {
            yylex();
            printf("lines = %d, chars = %d\n",
                    num_lines, num_chars);
            return 0;
        }
    \end{lstlisting}
\end{frame}

\begin{frame}[allowframebreaks]
    \frametitle{Lista de Tokens reconhecidos}
    \begin{itemize}
        \item comando para o preprocessador: ``\#''(.*)
        \item palavras reservadas: ``if''$|$``else''$|$``const''$|$``for''$|$``while''$|$``struct''
        \item tipos de dados: ``int''$|$``float''$|$``double''$|$``char''
        %\item atribuição: "="|"+="|"-="|"*="|"/="|"\%="|"<<="|">>="|"&="|"^="|"|="
        %\item operador aritimetico: "+""+"?|"-""-"?|"/"|"*"|"sizeof"|"["{INTEGER_LITERAL}"]"
        %\item operador relacional: "&&"|"||"|"!"|("="|"!")"="|("<"|">")"="?
        %\item fim de expressão: ;
        %\item início de bloco: \{
        %\item fim de bloco: \}
        %\item abre parenteses: (
        %\item fecha parenteses: )
        %\item inteiro literal: {DIGIT}+
        %\item float literal {INTEGER_LITERAL}"."{INTEGER_LITERAL}
        %\item string literal: \"[^\\\n\"]+\"
        %\item char literal: \'\\?.\'
        %\item identificador: ({LETTER})({ALPHA_NUM}|_)*
    \end{itemize}
\end{frame}

\begin{frame}
    \frametitle{Mão na massa}
    \includegraphics[width=\textwidth]{pizza.png}
    \includegraphics[width=\textwidth]{rolo.png}
    \includegraphics[width=\textwidth]{burrito.png}
\end{frame}

\section{Conclusão}
\begin{frame}
    \frametitle{Agradecimentos}
    \centering
    \Huge{Perguntas?}
    \includegraphics[width=\textwidth]{alerta.png}
    \includegraphics[width=\textwidth]{perigo.png}
    \includegraphics[width=\textwidth]{eletricidade.png}
    \Huge{Obrigado pela atenção}
\end{frame}

\section{Referências}\label{Referências}
\begin{frame}[allowframebreaks]
    \frametitle{Referências} 
    \bibliography{ref}
    \bibliographystyle{abbrv} % funciona
\end{frame}

\end{document}
